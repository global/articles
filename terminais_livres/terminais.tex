\documentclass[utf8,9pt]{beamer} 
\usepackage[brazil]{babel}
\usepackage{color}
\usepackage{lmodern} % fontes modernas
\usepackage[utf8]{inputenc}
\usepackage{listings}
\usepackage{pgfpages}
\usepackage{mathptmx}
\usepackage{helvet}

%\pgfpageuselayout{resize to}[a4paper,border shrink=5mm,landscape]
\title{Terminais "leves" X Multiterminais}
\author{Anderson Goulart - global@codekab.com}
\institute{uLevel.com}

\usetheme{Warsaw}

\pgfdeclareimage[height=4.5cm]{global}{figuras/global}
\pgfdeclareimage[height=5.0cm]{multiterminais}{figuras/multiterminais}
\pgfdeclareimage[height=3.5cm]{pc-tras}{figuras/pc-tras}
\pgfdeclareimage[height=3.5cm]{pc-frente}{figuras/pc-frente}
\pgfdeclareimage[height=4.0cm]{ltsp}{figuras/ltsp}
\pgfdeclareimage[height=4.0cm]{drbl}{figuras/drbl}
\pgfdeclareimage[height=5.0cm]{drblserver}{figuras/drblserver}
%\logo{\pgfuseimage{logo}}

\usecolortheme[RGB={1,152,183}]{structure}

\setbeamertemplate{footline}
{
	\insertframenumber
}


\begin{document}

	\begin{frame}
		\titlepage
	\end{frame}

	  \begin{frame}
		  \frametitle{Objetivos:}
		  \begin{itemize}
			  \item Multiterminais
			  \item Linux Terminal Server Project
			  \item Diskless Remote Boot in Linux (DRBL)
			  \item Projetos na área
			  \item Custos envolvidos
			  \item Apresentação de um terminal em funcionamento
		  \end{itemize}
	  \end{frame}

	  \begin{frame}
		  \begin{center}
			  \Huge{Multiterminais}
			  \begin{figure}
				  \pgfuseimage{multiterminais}
			  \end{figure}
		  \end{center}
	  \end{frame}

	  \begin{frame}
		  \frametitle{Multiterminais - introdução}
		  Os multiterminais funcionam da seguinte maneira:
		  \begin{itemize}
			  \item 1 servidor
			  \item 2 ou mais placas de vídeo
			  \item 2 ou mais teclados
			  \item 2 ou mais mouses
		  \end{itemize}
	  \end{frame}

	  \begin{frame}
		  \frametitle{Multiterminais - características}
		  Vantagens:
		  \begin{itemize}
			  \item Compartilhamento de recursos (home, software, etc)
			  \item Instalação de software centralizada no servidor
			  \item Não necessita dos recursos da rede
			  \item Economia na compra dos equipamentos
		  \end{itemize}

		  Desvantagens:
		  \begin{itemize}
			  \item Não funciona com mouse/teclado diferente de USB
			  \item O número de terminais depende da quantidade de placas de vídeo suportadas pelo seu computador
			  \item A distância é limitada pelos cabos ligados ao servidor
			  \item Difícil configuração
			  \item Drivers não estão 100\%
		  \end{itemize}
	  \end{frame}

	  \begin{frame}
		  \frametitle{Multiterminais - tecnologias}
		  Quais tecnologias existem para esse tipo de solução?
		  \begin{itemize}
			  \item Xephyr: um X server que roda dentro de um X server
			  \item Multiseat Xgl: um Xserver otimizado para aceleração 3d
			  \item Xorg + GDM: X server tradicional com configurações personalizadas
		  \end{itemize}
	  \end{frame}

	  \begin{frame}
		  \frametitle{Multiterminais - tecnologias}
		  Qual a idéia de implantação? \\
		  Mexer no xorg.conf
		  \begin{small}
		  \begin{block}{xorg.conf}
			  Section "InputDevice" \\
			  Identifier "Mouse0"\\
			  Driver "mouse" \\
			  Option "Protocol" "ImPS/2" \\
			  Option "ZAxisMapping" "4 5" \\
			  Option "Device" "/dev/input/mouse0" \\
			  EndSection \\
			  Section "InputDevice"\\
			  Identifier "Mouse1"\\
			  Driver "mouse"\\
			  Option "Protocol" "ImPS/2"\\
			  Option "ZAxisMapping" "4 5"\\
			  Option "Device" "/dev/input/mouse1"\\
			  EndSection
			  ...
			  Section "ServerLayout"\\
			  Identifier "Layout1"\\
			  Screen "Screen1" 0 0\\
			  InputDevice "Keyboard1" "CoreKeyboard"\\
			  InputDevice "Mouse1" "CorePointer"\\
			  EndSection
		  \end{block}
		  \end{small}
	  \end{frame}

	  \begin{frame}
		  \frametitle{Multiterminais - tecnologias}
		  E mexer no gdm.conf
		  \begin{block}{gdm.conf}
		  $[$servers$]$\\

		  0=Standard0\\
		  1=Standard1\\

		  $[$server-Standard0$]$\\
		  name=Standard server\\
		  command=/usr/X11R6/bin/X0 -sharevts -layout Layout0 \# esse X0 é um link simbólico para o X\\
		  flexible=false\\

		  $[$server-Standard1$]$\\
		  name=Standard server\\
		  command=/usr/X11R6/bin/X1 -sharevts -layout Layout1 \# nome do layout do xorg.conf;\\
		  flexible=false
		  \end{block}
	  \end{frame}

	  \begin{frame}
		  \frametitle{Multilinux}
		  No Brasil, temos um grupo que trabalha no desenvolvimento das tecnologias dos multiterminais: o \textbf{Multilinux}
		  \begin{block}{Implementação dos multiterminais}
		  O projeto MULTILINUX tem o objetivo de desenvolver tecnologia (scripts de instalação, automação do processo de instalação ou um LIVE CD) que permitam dividir um computador com GNU / Linux de maneira que vários usuário o utilizem ao mesmo tempo, em um ambiente gráfico, com sessões independentes. Uma excelente ferramenta para Inclusão Digital, ampliação de laboratórios, bibliotecas, quiosques de atendimento, telecentros e pequenas empresas.
		  \end{block}
	  \end{frame}

	  \begin{frame}
		  \frametitle{Multiterminais - foto}
		  
		  \begin{columns}
			  \column{.5\textwidth}
				  \begin{figure}
					  \pgfuseimage{pc-frente}
					  \caption{Multilinux - visão frontal}
				  \end{figure}

			  \column{.5\textwidth}
				  \begin{figure}
					  \pgfuseimage{pc-tras}
					  \caption{Multilinux - visão do gabinete}
				  \end{figure}
		  \end{columns}
	  \end{frame}

	  \begin{frame}
		  \begin{center}
			  \Huge{Linux Terminal Server Project}
			  \begin{figure}
				  \pgfuseimage{ltsp}
				  \caption{LTSP Official Site}
			  \end{figure}
		  \end{center}
	  \end{frame}


	  \begin{frame}
		  \frametitle{LTSP}
		  O Linux Terminal Server Project tem como objetivo proporcionar um servidor em ambientes Linux para alimentar os chamados \textit{thinclients}.\\
		  Um \textit{thinclient} ou terminal leve é um computador com recursos limitados: pouca memória, sem disco rígido, pouco processamento, etc.

	  \end{frame}

	  \begin{frame}
		  \frametitle{LTSP- características}
		  Vantagens:
		  \begin{itemize}
			  \item Compartilhamento de recursos (home, software, etc)
			  \item Instalação de software centralizada no servidor
			  \item É mais escalável que o modelo de multiterminais
			  \item Economia na compra dos equipamentos
			  \item Suporta maiores distâncias 
			  \item Projeto maduro
		  \end{itemize}

		  Desvantagens:
		  \begin{itemize}
			  \item A rede é um gargalo
			  \item Memória e CPU do servidor é outro gargalo
			  \item Dificuldade de funcionamento de todos os dispositivos: cdrom, impressoras, pendrives
			  \item Dificuldade de configuração: alta
		  \end{itemize}
	  \end{frame}

	  \begin{frame}
		  \frametitle{LTSP - tecnologias}
		  Quais tecnologias existem para esse tipo de solução?
		  \begin{itemize}
			  \item DHCP
			  \item DNS
			  \item TFTP
			  \item NFS
			  \item Sistemas de boot (PXE, Etherboot, etc)
			  \item Pacotes LTSP
		  \end{itemize}
	  \end{frame}

	  \begin{frame}
		  \frametitle{LTSP - versões}
		  Existem 2 grandes versões:
		  \begin{itemize}
			  \item 4.2 \\
				  Mais estável\\
				  Suporte a dispositivos com diversas gambiarras\\
				  LBE (LTSP Build Environment - chroot)
				  
			  \item 5.0 (Projeto MueKow)\\
				  Integração com a distro\\
				  Configuração simplificada\\
				  Redução do número de pacotes a ser distribuído\\
				  Novos mecanismos de comunicação com os dispositivos no cliente (ltspfs)\\
				  Controle personalizado do cliente (ltsp-build-client, init scripts)
		  \end{itemize}
	  \end{frame}

	  \begin{frame}
		  \frametitle{LTSP - Fotos}


	  \end{frame}

	  \begin{frame}
		  \frametitle{LTSP - Projetos}
		  Alguns projetos que utilizam LTSP:
		  \begin{itemize}
			  \item Projeto Casa Brasil
			  \item Projeto Petrobrás
			  \item UFMG
		  \end{itemize}
	  \end{frame}


	  \begin{frame}
		  \frametitle{LTSP - Fornecedores}
		  Existem diversos fornecedores de thinclients atualmente no Brasil:
		  \begin{itemize}
			  \item Tecnoworld
			  \item Connec
			  \item Samurai
			  \item Wise 
		  \end{itemize}
	  \end{frame}


	  \begin{frame}
		  \frametitle{LTSP - Custos}
		  \begin{itemize}
			  \item Valor do thinclient: R\$ 1.200,00
			  \item Servidor: R\$ 4.000,00
			  \item Total de equipamentos suportados por servidor: 20
		  \end{itemize}
	  \end{frame}

	  \begin{frame}
		  \begin{center}
			  \Huge{Diskless Remote Boot in Linux (DRBL)}
			  \begin{figure}
				  \pgfuseimage{drbl}
				  \caption{Logo DRBL}
			  \end{figure}
		  \end{center}
	  \end{frame}

	  \begin{frame}
		  \frametitle{DRBL - introdução}
		  O DRBL oferece uma estrutura semelhante ao LTSP, entretanto, ele criar imagens personalizadas para cada cliente. \\
		  A partir dessa idéia, os clientes são os responsáveis por executar as aplicações evitando o gargalo no servidor. \\
		  Os arquivos são compartilhados via NFS e os clientes podem armazenar arquivos locais, caso se tenha um disco.
	  \end{frame}

	  \begin{frame}
		  \frametitle{DRBL- características}
		  Vantagens:
		  \begin{itemize}
			  \item Compartilhamento de recursos (home, software, etc)
			  \item Instalação de software centralizada no servidor
			  \item É mais escalável que o modelo do LTSP
			  \item Suporta maiores distâncias 
			  \item Projeto maduro e com suporte outras features (clonezilla)
			  \item Configuração mais simples
		  \end{itemize}

		  Desvantagens:
		  \begin{itemize}
			  \item A rede é quase um gargalo
			  \item Gargalo no NFS
			  \item Os clientes tem que ser FAT (gordos)
			  \item Os clientes são o gargalo
			  \item Mais disco no servidor
			  \item O script de configuração é tosko!
		  \end{itemize}
	  \end{frame}

	  \begin{frame}
		  \frametitle{DRBL - tecnologias}
		  Quais tecnologias existem para esse tipo de solução?
		  \begin{itemize}
			  \item DHCP
			  \item DNS
			  \item TFTP
			  \item NFS
			  \item Sistemas de boot (PXE, Etherboot, etc)
			  \item Pacotes DRBL
		  \end{itemize}
	  \end{frame}

	  \begin{frame}
		  \frametitle{DRBL - Fotos}
		  \begin{center}
			  \begin{figure}
				  \pgfuseimage{drblserver}
				  \caption{Servidor com DRBL}
			  \end{figure}
		  \end{center}
	  \end{frame}

	  \begin{frame}
		  \frametitle{DRBL - Projetos}
		  Alguns projetos que utilizam DRBL:
		  \begin{itemize}
			  \item Escolas públicas 
			  \item UFMG
		  \end{itemize}
	  \end{frame}

	  \begin{frame}
		  \frametitle{DRBL - Custos}
		  \begin{itemize}
			  \item Valor do fatclient: R\$ 1.500,00
			  \item Servidor: R\$ 4.000,00
			  \item Total de equipamentos suportados por servidor: 40 (rede segmentada)
		  \end{itemize}
	  \end{frame}

	  \begin{frame}
		  \frametitle{Quadro comparativo}	
		  \begin{center}
			  \begin{table}[h!b!]
			  \begin{tabular}{llll}
			  \hline
			  . & Multilinux & LTSP & DRBL \\
			  \hline
			  Escalável & $--$ & $+-$ & $++$ \\
			  Custo & $+++$ & $+-$ & $++$ \\
			  Dificuldade de implantação & $+++$ & $++$ & $--$ \\
			  Gargalo na rede & $---$ & $+++$ & $+-$ \\
			  Gargalo no disco & $++$ & $+-$ & $++$ \\
			  Gargalo no cpu & $+++$ & $+++$ & $+-$\\
			  Maturidade do projeto & $--$ & $++$ & $++$\\
			  \hline
			  \end{tabular}
			  \caption{Comparação entre as tecnologias}
			  \end{table}
		  \end{center}
	  \end{frame}

	  \begin{frame}
		  \begin{center}
			  \Huge{Perguntas?}
		  \end{center}
	  \end{frame}

  \begin{frame}
	  \frametitle{Anderson Goulart}

	  \begin{columns}
		  \column{.5\textwidth}
		  Contato:
		  \begin{itemize}
			  \item Jabber: globalx@jabber.org
			  \item global@codekab.com
		  \end{itemize}
		  \column{.5\textwidth}
			  \begin{figure}
				  \pgfuseimage{global}
				  \caption{Global phishing!}
			  \end{figure}
	  \end{columns}
  \end{frame}

	\begin{frame}
		\frametitle{Referências}
		Sites:
		\begin{itemize}
			\item http://listas.softwarelivre.org/cgi-bin/mailman/listinfo/multilinux
			\item http://en.wikibooks.org/wiki/Multiterminal\_with\_Xephyr
			\item http://tecnoworld.tempsite.ws/Thin\_Client.asp
			\item http://www.ltsp.org/
			\item http://www.connec.com.br/en/p01a.asp
			\item http://www.jet.com.br/thinclientshop/
			\item http://www.neoware.com/
			\item http://drbl.sourceforge.net/
		\end{itemize}

	\end{frame}

  \begin{frame}
	  \frametitle{Licença}
	  "Terminais 'leves' X Multiterminais" por Anderson Goulart está licenciado sob a Creative Commons Atribuição-Compartilhamento pela mesma Licença 2.5 Brasil License.\\

	  http://creativecommons.org/licenses/by-sa/2.5/br/\\

	  \begin{itemize}
		  \item Fonte - http://github.com/global/articles
	  \end{itemize}
  \end{frame}



\end{document}

