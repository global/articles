\documentclass{beamer}

\usepackage[utf8]{inputenc}
%\usepackage{beamerthemesplit}
%\usepackage{beamerthemeCambridgeUS}
\usepackage{beamerthemeMadrid}

\title{Licenciamento de obras e distribuição de acervo na internet}
\author{Anderson Goulart - global@codekab.com}
\date{\today}

\begin{document}

\frame{\titlepage}

\section[Outline]{}
\frame{\tableofcontents}

\section{Conceitos iniciais}
\subsection{Legislação vigente}
\subsection{Das obras protegidas}
\subsection{Da autoria}
\subsection{Direitos morais e patrimoniais}
\subsection{Limitações e sanções previstas em lei}
\subsection{Convenção de Berna}

\section{Licenciamento livre de obras}
\subsection{Liberdades do software}
\subsection{Domínio Público}
\subsection{Creative Commons}
\subsection{GPL, FDL, MPL, etc}

\section{Referências Bibliográficas}

\frame
{
  \frametitle{Propriedade Intelectual}
  Divisão dentro de uma grande área do direito:
  \begin{itemize}
	\item Marca (trademark): um sinal, um nome, uma figura, uma identidade
	\item Patente: invenção com requisitos de novidade, atividade inventiva e aplicação industrial
 	\item Direito autoral (copyright): sobre as obras intelectuais, artísticas, científicas
  \end{itemize}
}

\frame
{
	\frametitle{Legislação vigente}
	\begin{itemize}
		\item Lei 9609/98 - Direitos autorais de programa de computador
		\item Lei 9610/98 - Direitos autorais das outras obras
		\item Lei 9279/96 - Propriedade industrial
		\item Lei 9456/97 - Cultivares
		\item Outras esparsas
	\end{itemize}

}

\frame
{
	\frametitle{Das obras protegidas}
	\begin{itemize}
		\item textos de obras literárias, artísticas ou científicas;
		\item as composições musicais, tenham ou não letra;
		\item as obras de desenho, pintura, gravura, escultura, litografia e arte cinética;
		\item os projetos, esboços e obras plásticas concernentes à geografia, engenharia, topografia, arquitetura, paisagismo, cenografia e ciência;
		\item as obras audiovisuais, sonorizadas ou não, inclusive as cinematográficas;
		\item os programas de computador;
	\end{itemize}
}

\frame
{
	\frametitle{O que NÃO é protegido}
	\begin{itemize}
		\item as idéias, procedimentos normativos, sistemas, métodos, projetos ou conceitos matemáticos como tais;
		\item os esquemas, planos ou regras para realizar atos mentais, jogos ou negócios;
		\item os formulários em branco para serem preenchidos por qualquer tipo de informação, científica ou não, e suas instruções;
		\item os nomes e títulos isolados;
		\item {\em o aproveitamento industrial ou comercial das idéias contidas nas obras.}
	\end{itemize}
}


\frame
{
	\frametitle{Da autoria}
	\begin{itemize}
		\item Autor é a pessoa física criadora de obra literária, artística ou científica.
		\item Para se identificar como autor, poderá o criador da obra literária, artística ou científica usar de seu nome civil, completo ou abreviado até por suas iniciais, de pseudônimo ou qualquer outro sinal convencional.
	\end{itemize}
}


\frame
{
	\frametitle{Direitos Morais}
	\begin{itemize}
		\item o de reivindicar, a qualquer tempo, a autoria da obra;
		\item de ter seu nome, pseudônimo ou sinal convencional indicado ou anunciado, como sendo o do autor, na utilização de sua obra;
		\item o de conservar a obra inédita; 
		\item o de modificar a obra, antes ou depois de utilizada;
		\item {\em Os direitos morais do autor são inalienáveis e irrenunciáveis.}
	\end{itemize}
}


\frame
{
	\frametitle{Dos direitos patrimoniais}
	\begin{itemize}
		\item Cabe ao autor o direito exclusivo de utilizar, fruir e dispor da obra literária, artística ou científica.
		\item Depende de autorização
		\begin{itemize}
			\item a reprodução parcial ou integral;
			\item a edição;
			\item a tradução para qualquer idioma;
			\item a utilização, direta ou indireta, da obra literária, artística ou científica
		\end{itemize}
		\item Ninguém pode reproduzir obra que não pertença ao domínio público, a pretexto de anotá-la, comentá-la ou melhorá-la, sem permissão do autor.
	\end{itemize}
}

\frame
{
	\frametitle{Dos direitos patrimoniais}
	\begin{itemize}
		\item Os direitos patrimoniais do autor perduram por setenta anos contados de 1 de janeiro do ano subseqüente ao de seu falecimento, obedecida a ordem sucessória da lei civil.
		\item O prazo de proteção aos direitos patrimoniais sobre obras audiovisuais e fotográficas será de setenta anos, a contar de 1 de janeiro do ano subseqüente ao de sua divulgação.
	\end{itemize}
}

\frame
{
	\frametitle{Limitações de uso}
	\begin{itemize}
		\item Não constitui ofensa aos direitos autorais:
		\begin{itemize}
			\item a reprodução, em um só exemplar de pequenos trechos, para uso privado do copista, desde que feita por este, sem intuito de lucro;
			\item a representação teatral e a execução musical, quando realizadas no recesso familiar ou, para fins exclusivamente didáticos, nos estabelecimentos de ensino, não havendo em qualquer caso intuito de lucro
			\item a citação em livros, jornais, revistas ou qualquer outro meio de comunicação
			\item a reprodução na imprensa diária, em diários ou periódicos de notícias ou discursos pronunciados publicamente
		\end{itemize}
	\end{itemize}
}

\frame
{
	\frametitle{Sanções}
	\begin{itemize}
		\item Quem editar obra literária, artística ou científica, sem autorização do titular, perderá para este os exemplares que se apreenderem e pagar-lhe-á o preço dos que tiver vendido.
		\item Não se conhecendo o número de exemplares que constituem a edição fraudulenta, pagará o transgressor o valor de três mil exemplares, além dos apreendidos.
		\item Art. 12. Violar direitos de autor de programa de computador:
		\begin{itemize}
			\item Pena - Detenção de seis meses a dois anos ou multa.
		\end{itemize}
	\end{itemize}
}

\frame
{
	\frametitle{Convenção de Berna}
	\begin{itemize}
		\item Convenção entre países signatários para proteção dos direitos do autor.
		\item Mínimo de 50 anos após a morte do autor.
	\end{itemize}
}

\frame
{
	\begin{center}
	Licenciamento Livre de Obras
	\end{center}
}

\frame
{
	\frametitle{Liberdades de Software}
	\begin{itemize}
		\item <1->Liberdade 0: A liberdade de executar o programa, para qualquer propósito
		\item <2->Liberdade 1: A liberdade de estudar como o programa funciona, e adaptá-lo para as suas necessidades. Acesso ao código-fonte é um pré-requisito para esta liberdade
		\item <3->Liberdade 2: A liberdade de redistribuir cópias de modo que você possa ajudar ao seu próximo
		\item <4->Liberdade 3: A liberdade de aperfeiçoar o programa, e liberar os seus aperfeiçoamentos, de modo que toda a comunidade se beneficie. Acesso ao código-fonte é um pré-requisito para esta liberdade
		\item <5->Isso representa o conceito de COPYLEFT
	\end{itemize}
}

\frame
{
	\frametitle{Domínio Público}
	\begin{itemize}
		\item Direitos econômicos não são de exclusividade de nenhum indivíduo ou entidade.
		\item Qual é o problema com ele? 
	\end{itemize}
}

\frame
{
	\frametitle{Creative Commons}
	\begin{itemize}
		\item Organização sem fins licrativos para ajudar os autores a licenciar suas obras.
		\item Tipos de licença
		\begin{itemize}
			\item Attribution-Noncommercial-Share Alike 2.5 Generic
			\item Atribution-Share Alike
			\item Atribution-No derivative works
			\item Atribution
		\end{itemize}
	\end{itemize}
}

\frame
{
	\frametitle{Outras licenças}
	\begin{itemize}
		\item GPL
		\item BSD
		\item LGPL
		\item AGPL
		\item FDL, MPL, etc
	\end{itemize}
}

\frame
{
	\frametitle{Passos para licenciar uma obra}
	\begin{itemize}
		\item Escolha a licença (o site creativecommons.org.br lhe ajuda)
		\item Indique na obra a licença escolhida em alguma nota de copyright (ou copyleft)
		\item Indique o nome do autor, ano e data de publicação, título da obra
	\end{itemize}
}

\frame
{
	\frametitle{Referências Bibliográficas}
	\begin{itemize}
		\item Lei 9610 - Direitos autorais\\
		http://www.planalto.gov.br/ccivil\_03/leis/l9610.htm
		\item Lei 9609 - Propriedade intelectual de programa de computador\\
		http://www.planalto.gov.br/ccivil\_03/LEIS/L9609.htm
		\item INPI\\
		http://www.inpi.gov.br/
		\item Domínio Público\\
		http://www.dominiopublico.gov.br
		\item Creative Commons\\
		http://www.creativecommons.org.br
	\end{itemize}

}

\frame
{
	\frametitle{Dúvidas}
	\begin{itemize}
		\item Anderson Goulart - global@codekab.com
	\end{itemize}
	

}

\end{document}

